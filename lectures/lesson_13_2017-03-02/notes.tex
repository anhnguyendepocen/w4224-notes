\documentclass{article}

\usepackage{amsmath, amssymb, amsfonts, amsthm, booktabs, verbatim, mathtools, listings, xcolor}

%\numberwithin{equation}{section}
\newtheorem{theorem}{Theorem}
\newtheorem{lemma}{Lemma}
\newtheorem{proposition}{Proposition}
\newtheorem{definition}{Definition}
\newtheorem{problem}{Problem}

% Surrounding angular brackets
\newcommand{\surang}[1]{\langle #1 \rangle}
\makeatletter
\newcommand{\@giventhatstar}[2]{#1\,\middle|\,#2}
\newcommand{\@giventhatnostar}[3][]{#1(#2\,#1|\,#3#1)}
\newcommand{\giventhat}{\@ifstar\@giventhatstar\@giventhatnostar}
\makeatother
\newcommand{\probof}[1]{\text{Pr}\left( #1 \right)}
\newcommand{\pdens}[1]{\text{p}\left( #1 \right)}
\newcommand{\variance}[1]{\text{Var}\left( #1 \right)}

\lstset{ %
  language=R,                     % the language of the code
  basicstyle=\footnotesize,       % the size of the fonts that are used for the code
  numbers=left,                   % where to put the line-numbers
  numberstyle=\tiny\color{gray},  % the style that is used for the line-numbers
  stepnumber=1,                   % the step between two line-numbers. If it's 1, each line
                                  % will be numbered
  numbersep=5pt,                  % how far the line-numbers are from the code
  backgroundcolor=\color{white},  % choose the background color. You must add \usepackage{color}
  showspaces=false,               % show spaces adding particular underscores
  showstringspaces=false,         % underline spaces within strings
  showtabs=false,                 % show tabs within strings adding particular underscores
  frame=single,                   % adds a frame around the code
  rulecolor=\color{black},        % if not set, the frame-color may be changed on line-breaks within not-black text (e.g. commens (green here))
  tabsize=2,                      % sets default tabsize to 2 spaces
  captionpos=b,                   % sets the caption-position to bottom
  breaklines=true,                % sets automatic line breaking
  breakatwhitespace=false,        % sets if automatic breaks should only happen at whitespace
  title=\lstname,                 % show the filename of files included with \lstinputlisting;
                                  % also try caption instead of title
  keywordstyle=\color{blue},      % keyword style
  commentstyle=\color{red},   % comment style
  stringstyle=\color{black},      % string literal style
  escapeinside={\%*}{*)},         % if you want to add a comment within your code
  morekeywords={*,...}            % if you want to add more keywords to the set
} 

\begin{document}

AIC favors full pooling over no pooling, but has nothing to say about the hierarchical model. (We're at 3 down and 9 to go).

Deviance Information Criterion:

There are two problems with AIC.
\begin{enumerate}
	\item 
		AIC is not Bayesian. It doesn't account for prior information.
	\item
		The number of parameters is not well-defined.
\end{enumerate}

For example in our 8 schools example, we have unpooled $\tau = \infty$ which
implies that $k = 8$, the fully pooled $\tau = 0$, which implies $k = 1$, and
the hierarchical model $\tau \propto 1$ which implies $1 < k < 8$.

To solve the first problem we replace $\hat{\theta} _{\text{mle}}$ with $\hat{\theta}_{\text{Bayes}} = E\left( \giventhat*{\theta}{y} \right)$

To solve the second problem we replace $k$ with a data-based biased correction, i.e. $2 \left( \log \pdens{\giventhat*{y}{\hat{\theta}_{\text{Bayes}}}} - E_{\text{post}}\left( \log \pdens{\giventhat*{y}{\theta}} \right) \right)$.

In our particular case we have
\begin{equation}
	\theta _{j, \text{Bayes}} = E\left( \giventhat*{\theta _j}{y} \right) = y_j = \hat{\theta} _{j, \text{mle}}
\end{equation}`
so
\begin{align}
	\log \pdens{\giventhat*{y}{\hat{\theta} _\text{Bayes}}} &= \prod _{j = 1} ^J \frac{1}{\sigma _j \sqrt{2 \pi}} e ^{-\frac{1}{2 \sigma _j} \left( y_j - y_j \right)}
\end{align}

In the case of DIC we have $p_{DIC} = 2 \variance{\giventhat*{\log \pdens{\giventhat*{y}{\theta}}}{y}}$.
Note that this ends up being the same as AIC.

After eliding a bunch of calculations, we get that $DIC = -2 \log \pdens{\giventhat*{y}{\hat{\theta} _\text{Bayes}}} + 2 p_{\text{DIC}} = -2\left( -\frac{J}{2} \log (2 \pi) - \sum \log \sigma _j \right) + 2 \cdot 8 = 54.6 + 16 = 70.6$ which is the same answer as AIC.

Note that DIC should agree with AIC since the two problems that DIC fixes weren't problems for the model.

So in the case that we have complete pooling, namely that $\tau = 0$ we can 

\end{document}
